%\usepackage{enumitem,amssymb} - pakiet odpowiedzialny za kwadraciki do zakreślania
%\usepackage{multicol} - pakiet odpowiedzialny za podział tekstu na kolumny

%\newlist{checkboxlist}{itemize}{2}
%\setlist[checkboxlist]{label=$\square$}

\chapter{Coping with Oppression}
Artist: Mohammed Fayaz

\newpage

Artist: Jacks McNamara
How do you cope with the impact of oppression?

\begin{multicols}{2}
\begin{checkboxlist}
Friends
\item Draw
\item Exercise
\item Martial arts
\item Eating healthy
\item Nature
\item Yoga
\item Icarus
\item Creative expression
\item Writing
\item Bicycling
\item Being part of a community
\item Stretching
\item Learn about it
\item Fighting oppression!
\item Pray for guidance
\item Activism!
\item Not giving up
\item Meditation
\item Knowing that I am not broken!
\item Laughter. Always laughter
\item Peer counseling
\item Social networking
\item Owning my opinions
\item Sing
\item Humor
\item Intellectualizing it
\item Breaking stereotypes
\item Compassion
\item Doing things that make me happy
\item Helping others
\item Speaking my mind
\item Reading
\item Positive relationships
\item Playing with a pet
\item Admitting when I’m not okay
\item Going to the gym
\item Taking a stand
\item Spiritual practice
\item Educating myself
\item Therapy
\item Limiting exposure to oppressor
\item Homeopathics
\item Strengthening my personal love
\item Bodywork
\item I developed talking points
\item Cuddling
\item Advocacy
\item Routine
\item Being open about my challenges
\item Talking openly about it
\item Reading empowering statements, essays, and poems
\item Sculpture
\item Dance
\item Film
\item Photography
\item Solidarity with others
\item Art
\item Music
\item Picking my battles
\item Organising
\item Educating others
\item Learning to say “no”
\item Unafraid to tell the “truth” as I see it
\item Power of Positive Thinking & Action
\item Communicating to others
\item Having a chosen family
\item Studying
\item Contextualizing behavior within systemic violence so it is less shameful
\item Thinking
\item Seeing a counselor.
\item Trying to take care of myself
\item Allowing ourselves to cry
33\item Reading about other people who experience oppression in order not to feel alone
\item Reiki
\item Going to rallies
\item Exercise
\item Writing
\item Being in nature
\item Listening to cheerful music
\item Caring for animals
\item Reminding ourselves that not everyone will treat us poorly
\item Nurturing people
\item Detox
\item Being a friend and teacher for others
\item Reminding ourselves that the oppressors are at fault, not us
\item Writing and reading
\item Acknowledge the feeling, experiencing it in our bodies, and then, after a time, try to let it go
\item Art
\item Take deep breaths
\item Venting
\item Turning feelings into action: finding a healthy venue for all the passion and emotional work that needs to be done
\item Reminding ourselves that we love life
\item Escapism into a book or a TV show is nice
\item Doing more rather than just existing
\item Meditate on simplicity and non-violent solutions
\item Reaching out for support
\item Expressing the inexpressible
\item Spiritual practices
\item Engage in self care
\item Deep breaths
\item Taking a stand
\item Developing a good relationship with a trusted healthcare provider
\item Allow ourselves to take a day or two to recover
\item Remembering that life is not a race
\item Self-talk
\item Exercise
\item Sleep
\end{checkboxlist}
\end{multicols}

\newpage

Artist: Eddy Falconer
How else can you cope with the impact of oppression?

\newpage

How do you navigate triggering situations?
Artist: Jess Rankine
\begin{checkboxlist}
\item Try to stay cool
\item Detach as quickly as possible
\item Listen to peaceful music
\item Paint
\item Advise the person who triggered the sensation that you need to take a moment and, say, get a glass of water
\item Reach out to others if you feel too overwhelmed
\item Let people know how they are affecting you
\item Disengage or go for a walk
\item Recognize where the anxiety is coming from
\item Stay in your comfort zone with the people you know until you find someone trusted you can confide in
\end{checkboxlist}

What helps you navigate triggering situations?
How can others help you?

\newpage

How can people help you?
How can we help each other?

\begin{multicols}{2}
\begin{checkboxlist}
\item Don’t tell me to “Snap out of it” or “It will be alright”
\item Make me some tea
\item Bring me flowers
\item Don’t ignore me
\item Show me you love me
\item Normalize my feelings
\item Join in on the action
\item Validate me
\item Believe
\item Name the oppression
\item Make connections with others who have similar experiences
\item Validate my experience
\item Normalize
\item Listen to me
\item Say it is not okay
\item Don’t try to fix me
\item Help me up to a better plane of existence
\item Be present
\item Run me a bath
\item Hear and acknowledge the experience
\item Getting me comfy clothes to change into
\item Distract me
\item Tell me a joke or share funny stories
\item Watch a movie with me
\item Talk to me about superficial topics, like celebrities or popular culture
\item Don’t discourage my dreams
\item Believe I am capable of anything despite my struggles and obstacles
\item Educate yourself
\item Help me focus on the things I like
\item Acknowledge that you are capable of racist acts without being racist
\item Appreciate my voice
\item Don’t tell me how I feel
\item Be an ear, even if you don’t agree with me
\item Speak about positive things
\item Engage in intellectual debate
\item Invite me to get together
\item Don’t shoot down my ideas
\item Come visit
\item Treat me like an adult
\item Bring over a treat
\item Tell me that you love me regardless of what happens to me
\item Remembering that it takes me time to heal
\item Stay with me
\item Don’t pressure me
\item Give me space
\item Accept my feelings
\item Be aware
\item Don’t give unsolicited advice
\item Have patience
\item Remind me of things that have helped me in the past
\item Be understanding
\item Show compassion
\item Don’t say, “I know exactly how you feel”
\item Remind me of my skills
\item Hug me
\item Help me use some my skills
\item Believe in what I say
\item Be there for me
\item Hold my hand
\item Don’t abandon me
\item Offer to call my therapist for me
\item Help me by reducing pressure
\item Offer to run errands
\end{checkboxlist}
\end{multicols}

In what other ways can people help you?
What is helpful? What isn’t?
