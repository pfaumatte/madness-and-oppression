\chapter{Transforming our Communities}


\begin{figure}[h]
	\centering
	\includegraphics[height=12cm]{TeX_files/4-0.png}
	\caption{Artist: Anastasia Keck}
	\label{2-0}
\end{figure}

\noindent\textcolor{ProcessBlue}{\textbf{\LARGE{Where can we begin to address oppression in our communities?}}}\\

We can begin by talking about it. There are many benefits to talking about
oppression in our communities. These include:

\begin{checkboxlist}
	\item When you understand and acknowledge what is really happening and why, you can address it directly and begin to actually problem-solve effectively. There are no benefits - only costs - to inequality.
	\item We need to look at all the ways people are hurt by oppression because that could lead to people telling the truth about how oppressive our society is, and that could hopefully lead to positive changes.
	\item People would know that they’re not alone.
	\item If we don’t explore it, we leave it up to the authorities, and their perspectives are important but limited. If we do explore it, we do have a chance to reduce the amount of madness and oppression and to improve personally and communally.	
	\item Oppression must be understood in order to counter it. In a community setting, the community as a whole must understand how oppression is generated by groups and directed against individual members of the community.
	\item Talking about these issues in a safe way can only have a positive effect on everyone.
	\item It can help us bond: we can learn greater depth in how to support people who are challenged by emotions or behaviors that interfere with their life.
	\item More compassion could be accessed through exploring oppression. People tend to dehumanize what they do not understand.
	\item The community will benefit because they will become more aware of their suffering and may reach out from the shadows to join programs or start new ones so oppression will become a nonword.
	\item Exposing oppression helps the individual get a sense of justice and it helps the community to see the oppression and stop it.
\end{checkboxlist}

\noindent\textcolor{ProcessBlue}{\textbf{\LARGE{How can you begin to address oppression in your community?}}}\\

\noindent\rule{\textwidth}{1pt}\\
\noindent\rule{\textwidth}{1pt}\\
\noindent\rule{\textwidth}{1pt}\\
\noindent\rule{\textwidth}{1pt}\\
\noindent\rule{\textwidth}{1pt}\\
\noindent\rule{\textwidth}{1pt}\\
\noindent\rule{\textwidth}{1pt}\\
\noindent\rule{\textwidth}{1pt}\\\\



\begin{figure}[h]
	\centering
	\includegraphics[height=12cm]{TeX_files/4-1.png}
	\caption{Artist: Till Krech}
	\label{2-0}
\end{figure}

\noindent\textcolor{ProcessBlue}{\textbf{\LARGE{Steps we can take to mitigate oppression include:}}}\\

\begin{multicols}{2}
	\begin{checkboxlist}
		\item Do the work. Educate yourself. People need to educate themselves, speak up, be okay with feeling uncomfortable
		\item Join social causes. Contribute work in your own way, whether marching in the streets or stuffing envelopes
		\item Sign petitions
		\item Speak at City Council meetings
		\item Educate your children
		\item End the silence. Talk about it. Name it. Show support
		\item Raise awareness: talk to friends
		\item Be outspoken
		\item Confront ‘isms’ as they arise
		\item Be a good ally
		\item Help show your community that we are here
		\item Teach children that homeless people have feelings and that LGBT parents are just like theirs
		\item People could educate themselves about PTSD and understand that it’s a part of who I am, and they’ll appreciate me, and other sufferers, all the more for it
		\item Prevent bullying
		\item ake actions together and plan actions in groups. They could educate themselves on oppression, privilege, racism, cultural competency, and so forth
		\item Support families in your communities. Make establishments intergenerational to include children and the elderly
		\item Demand that the homeless be allowed to sleep in their homeless camps.They could build a support system for activists and  fighting oppression and encourage their efforts and their voice and their struggle
		\item I believe the first step is to cultivate an open mind, to believe that we are all capable of healing and change
		\item Participate in letter-writing campaigns to companies and complain
		\item Accept that there are people like me and that I desire to help them
		\item Accept that I am able to do art and poetry because of my life’s journey
		\item Go to rallies and marches to support others
		\item Sign petitions and email politicians
		\item Display anti-racist stickers
		\item Talk to others to raise awareness
		\item Create an inner peace and that is indestructible
	\end{checkboxlist}
\end{multicols}

\noindent\textcolor{ProcessBlue}{\textbf{\LARGE{Steps we have taken to mitigate oppression include:}}}\\

\begin{multicols}{2}
	\begin{checkboxlist}
		\item I now have a yearning to change the situations and bring others into the light of understanding, that life is beautiful and precious	
		\item I believe the feelings of oppression have given me immense compassion and understanding
		\item I have written and directed short plays, and drawn and painted lovely pictures. I could not have done those things had I not been in such pain
		\item It’s made me stronger in my faith
		\item I’ve made friends and deepened relationships
		\item I’ve done spoken word
		\item I’ve made zines
		\item I helped start or participated in several anti-oppression movements
		\item Sometimes I think about how many people may feel similar to the way I feel, and it inspires me to write zines or fiction, thinking there may be this unknown-to-me audience that might take value in my words
		\item I found acceptance in a community that values people for who they are
		\item I found a sense of liberation in beginning to accept myself completely
		\item I’m trying to engage other users/survivors in creating a community that focuses on alternative conceptualizations of mental health and well-being and on political action to bring about systemic change
		\item The only “cure” for powerlessness is social and political activism - my only concern is about caring for the ones who get burnt out and cannot find a caring community, and that is something we need to work on as a community
		\item I have learned how to strengthen my chakras and my aura so as not to be bombarded with other peoples negative energy		
		\item I prioritize my own health, safety, and self-expression
		\item I try to be good to others, to be the best person I can be to make the world more tolerable
		\item I feel a strong need to change it all, or at least stir it up with a big spoon
		\item I tend to philosophise at length in the comfort of the bedroom, take walks outside, and write poetry to do justice to such sentiments and not get totally burnt out
		\item I organize to build power with our people to overcome our oppressions
	\end{checkboxlist}
\end{multicols}

\noindent\textcolor{ProcessBlue}{\textbf{\LARGE{What other ideas do you have for transforming your community?}}}\\

\noindent\rule{\textwidth}{1pt}\\
\noindent\rule{\textwidth}{1pt}\\
\noindent\rule{\textwidth}{1pt}\\
\noindent\rule{\textwidth}{1pt}\\
\noindent\rule{\textwidth}{1pt}\\
\noindent\rule{\textwidth}{1pt}\\
\noindent\rule{\textwidth}{1pt}\\
\noindent\rule{\textwidth}{1pt}\\\\
